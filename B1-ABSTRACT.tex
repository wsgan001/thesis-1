%\vspace{-2.5cm}
\chapter*{\zihao{2}\heiti{摘~~~~要}}
%\vspace{-1cm}
随着信息技术和互联网的发展,网民用户和网络产品数量成爆炸式增长,用户从信息匮乏时代进入信息过载时代。个性化推荐系统对用户行为和企业商品特性数据建模,为用户提供满足他们兴趣和需求的信息,同时为企业推广提供目标客户。现代互联网服务提供商,例如淘宝等在线购物网站、爱奇艺等在线视频网站、大众点评等生活信息服务网站,提供大量商品给用户消费,让用户给商品评分以及使用标签描述商品。针对以上用户行为数据,本文以矩阵分解相关理论为基础,建立模型以解决推荐系统中的三个典型任务:(1)在隐式反馈数据上构建\textit{商品推荐}模型,推荐用户感兴趣的商品;(2)利用显式评分数据建立\textit{评分预测}模型,预测用户对商品的喜好值;(3)利用显式标签数据建立\textit{标签推荐}模型,方便用户输入来描述商品属性,帮助推荐系统良性循环。本文的研究问题和技术贡献总结如下:

\begin{itemize}
	\item[1.] 基于加权局部矩阵分解的\textit{商品推荐}:现有的针对隐式反馈数据的矩阵分解模型往往只从数据的全局信息出发,忽略了数据之中的局部信息。为了利用隐式反馈数据的局部信息,本文提出了一种加权局部矩阵分解模型进行商品推荐,并为该模型设计了高效的子矩阵选择算法和改进的交替最小二乘参数优化算法,对用户和商品的局部特性建模,同时缓解了数据稀疏性问题。真实数据上的实验结果表明该模型有较优的推荐效果,并验证了考虑隐式反馈数据的局部信息有助于商品推荐。
	
	\item[2.] 基于多主题矩阵分解的\textit{评分预测}:为克服现有工作中针对显式评分数据局部信息建模的不可解释性和目标函数的不一致性,本文提出了多主题矩阵分解模型。它结合主题模型和概率矩阵分解模型,利用主题模型建模数据局部信息和矩阵分解来刻画用户和商品的局部内在特征。本文使用贝叶斯方法建模主题矩阵分解模型,使得模型只需少量的经验设置参数以得到更高的推荐准确率。实验结果说明该模型在评分预测中优于其他局部矩阵分解模型,并对局部建模信息具有一定的可解释性。
	
	\item[3.] 时间感知的\textit{标签推荐}:为了利用标签数据中用户标注标签的时间信息,本文提出了时间感知的张量分解模型。该模型利用Hawkes时间点过程对用户使用标签的时间信息建模,并利用指数函数将Hawkes过程中的叠加形式转化为递归形式,使得计算用户当前时间对标签的喜好值只跟上次使用标签的时间有关,减少了大量的计算时间,并将其以权重的方式加入到逐对排序张量分解模型。实验结果表明该模型能够有效地利用时间信息,提高标签准确度,同时在冷启动问题上也有较好的表现,并且具有可接受的推荐新颖性。
\end{itemize}
\hspace{-0.5cm}
\sihao{\heiti{关键词:}} \xiaosi{矩阵分解,局部信息,主题模型,商品推荐,评分预测,标签推荐}
%额外空白页