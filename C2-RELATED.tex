\chapter{问题定义及研究现状}\label{chapter:relatedwork}
本章第一节首先对数据模型和要解决的查询问题进行了定义,然后从以下方面介绍了相关的研究工作:轨迹索引、轨迹降维、轨迹相似度度量、时间序列数据top-$k$查询,最后对本章内容进行小结。

\section{问题定义}\label{chapter-related-coll}
本文的研究目标是给定一条查询轨迹,从存储在若干远程结点中的轨迹中找出$k$条距离最短或相似度最高的轨迹。为此,本节首先介绍了轨迹数据模型,接着介绍了查询的定义。

\subsection{轨迹数据}
轨迹是描述移动对象移动行为的数据。通常来说,一条轨迹$T$可看做包含$n$个元素(即轨迹点)的有序序列。每个轨迹点$\vp$包含了时间、位置等维度的信息。因此轨迹可被形式化定义为如下:
\begin{define}[轨迹]
轨迹形式化表示为:$T=\{\bp_{0}, \bp_{1},\cdots, \bp_{n-1}\}$。
\begin{equation}
T=\{\bp_{0}, \bp_{1},\cdots, \bp_{n-1}\}
\end{equation}
其中$|T|=n$代表轨迹所包含的点数,即轨迹的长度。每个轨迹点$\bp$包含时间($t$)、位置($l$)等维度的信息。因而$|\bp|=d$称为轨迹的维度。此外轨迹中的点严格按时间升序排列,即$\forall i,j,0\le i\le j < n$则$\bp_{i}.t \le \bp_{j}.t$。
\end{define}

轨迹数据的来源多样且复杂。根据移动对象的划分可分为如下几类:
\begin{itemize}
	\item \textsf{人类活动轨迹数据:}该类数据分为主动式和被动式。主动式数据是人们主动利用移动定位设备分享或汇报自己的位置等信息。典型的有社交网络中的数据,用户提交位置获得服务的数据。被动式数据是人们无意间使用各种服务时所产生的轨迹数据。典型的有公交刷卡轨迹和手机的信令轨迹数据。

	\item \textsf{交通工具轨迹数据:}这类数据主要是交通工具使用车载GPS设备所产生的移动轨迹数据。例如,出租车、公交车的活动轨迹数据。
	
	\item \textsf{动物活动轨迹数据:}这类数据是为了研究动物生活、迁徙等行为和习惯而捕获的数据。
	
		\item \textsf{自然现象活动轨迹数据:}这类数据典型的有台风、冰山、海洋事件等的轨迹数据,用以探索自然现象的活动规律。
\end{itemize}

轨迹数据符合大数据时代的 3V 特征,即量大、实时、多样。轨迹数据采样由于受设备、采样频率等因素影响,数据质量较低且各个轨迹的采样间隔差异显著。
这些问题导致原始轨迹数据的可用性较低。因此,我们在进行轨迹数据分析前往往需要经过数据清理(data cleaning)、地图匹配(map mathching)、轨迹分段(trajectory segmentation)等预处理方式化为校准轨迹。校准轨迹数据能够通过数据管理技术进行轨迹索引以便有效地存取。因此,本文所处理的轨迹数据为预处理后的校准轨迹数据。这样的数据有如下特点:(i)采样频率一致;(ii)长度一致;(iii)位置精度高。
这为我们挖掘轨迹模式从而提炼有价值的知识提供了可靠保障。

\subsection{分布式k近邻轨迹查询}
轨迹数据往往是分布式采集并存储的。为此假设有$M$个远程结点,每个远程结点$i$包含轨迹数据集$TS_{i}$。那么整个分布式轨迹数据集$TS=\bigcup_{i=1}^{M} TS_{i}$。我们的目标是给定查询轨迹,从分布式存储的$TS$数据集中,找出与其距离最近的$k$条轨迹。下面我们将给出查询的形式化定义:
\begin{define}[分布式k近邻轨迹查询]
	该查询形式为query$({\cal Q}, TS,DM,k)$,其中$\cal Q$为给定查询轨迹,$TS$为分布式轨迹数据集, $DM$为距离度量准则以及$k$为返回结果集大小。查询的目标是返回满足如下条件的轨迹集$\cal S$:(1)${\cal S} \subseteq TS$;(2)$| {\cal S}|=k$;
	(3)$\forall {\cal C} \in {\cal S}, {\cal C}' \in {TS - \cal S}$,$DM({\cal Q},{\cal C}) \le DM({\cal Q},{\cal C}')$。
\end{define}

传统的集中式环境下$k$近邻轨迹查询相比,分布式场景下的查询不仅注重查询效率,而且尤其注重通信开销。这是由于分布式场景中,远程结点和协调者结点的带宽资源往往是有限的。高的通信开销,意味着用户可能要花费更多的金钱。因此,用户允许多花一点时间以达到降低通信开销的目的。

\section{轨迹降维}\label{sec-c2-reduction}
为降低数据传输开销,一个直观的想法就是先对轨迹进行降维,然后将降维后的数据发送给。

\section{轨迹索引}\label{sec-c2-index}


\section{轨迹相似度度量}\label{sec-c2-measures}

\section{时间序列top-$k$查询}\label{sec-c2-topk}






\clearpage
\phantom{s}
\clearpage